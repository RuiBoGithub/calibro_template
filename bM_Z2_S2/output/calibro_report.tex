\documentclass[a4paper]{report}\usepackage[]{graphicx}\usepackage[]{xcolor}
% maxwidth is the original width if it is less than linewidth
% otherwise use linewidth (to make sure the graphics do not exceed the margin)
\makeatletter
\def\maxwidth{ %
  \ifdim\Gin@nat@width>\linewidth
    \linewidth
  \else
    \Gin@nat@width
  \fi
}
\makeatother

\definecolor{fgcolor}{rgb}{0.345, 0.345, 0.345}
\newcommand{\hlnum}[1]{\textcolor[rgb]{0.686,0.059,0.569}{#1}}%
\newcommand{\hlstr}[1]{\textcolor[rgb]{0.192,0.494,0.8}{#1}}%
\newcommand{\hlcom}[1]{\textcolor[rgb]{0.678,0.584,0.686}{\textit{#1}}}%
\newcommand{\hlopt}[1]{\textcolor[rgb]{0,0,0}{#1}}%
\newcommand{\hlstd}[1]{\textcolor[rgb]{0.345,0.345,0.345}{#1}}%
\newcommand{\hlkwa}[1]{\textcolor[rgb]{0.161,0.373,0.58}{\textbf{#1}}}%
\newcommand{\hlkwb}[1]{\textcolor[rgb]{0.69,0.353,0.396}{#1}}%
\newcommand{\hlkwc}[1]{\textcolor[rgb]{0.333,0.667,0.333}{#1}}%
\newcommand{\hlkwd}[1]{\textcolor[rgb]{0.737,0.353,0.396}{\textbf{#1}}}%
\let\hlipl\hlkwb

\usepackage{framed}
\makeatletter
\newenvironment{kframe}{%
 \def\at@end@of@kframe{}%
 \ifinner\ifhmode%
  \def\at@end@of@kframe{\end{minipage}}%
  \begin{minipage}{\columnwidth}%
 \fi\fi%
 \def\FrameCommand##1{\hskip\@totalleftmargin \hskip-\fboxsep
 \colorbox{shadecolor}{##1}\hskip-\fboxsep
     % There is no \\@totalrightmargin, so:
     \hskip-\linewidth \hskip-\@totalleftmargin \hskip\columnwidth}%
 \MakeFramed {\advance\hsize-\width
   \@totalleftmargin\z@ \linewidth\hsize
   \@setminipage}}%
 {\par\unskip\endMakeFramed%
 \at@end@of@kframe}
\makeatother

\definecolor{shadecolor}{rgb}{.97, .97, .97}
\definecolor{messagecolor}{rgb}{0, 0, 0}
\definecolor{warningcolor}{rgb}{1, 0, 1}
\definecolor{errorcolor}{rgb}{1, 0, 0}
\newenvironment{knitrout}{}{} % an empty environment to be redefined in TeX

\usepackage{alltt}

\usepackage[tmargin=1in,bmargin=0.6in,lmargin=0.8in,rmargin=0.6in]{geometry}
%\usepackage[default]{sourcesanspro}
\usepackage[utf8]{inputenc}
\usepackage{graphicx}
\usepackage[section]{placeins}
\usepackage{enumitem}
\usepackage{caption}
\usepackage{booktabs}
\usepackage{float}
\usepackage{tabularx}
\usepackage{hyperref}
\usepackage{eso-pic}
\usepackage{fancyhdr}

\captionsetup[figure]{skip=1pt}
\captionsetup[table]{skip=1pt}

\hypersetup{
  colorlinks   = true, %Colours links instead of ugly boxes
  urlcolor     = blue, %Colour for external hyperlinks
  linkcolor    = blue, %Colour of internal links
  citecolor   = red %Colour of citations
}

\newcommand*{\thead}[1]{\multicolumn{1}{c}{\textbf{#1}}}
\renewcommand{\thesection}{\arabic{section}}
\renewcommand{\abstractname}{Executive Summary}



%include backgroung image on first page
\newcommand\BackgroundPic{%' >> "$report"
	\put(0,0){
		\parbox[b][\paperheight]{\paperwidth}{
			\vfill
			\centering
			\includegraphics[width=\paperwidth,height=\paperheight, keepaspectratio]{/opt/homebrew/lib/R/4.3/site-library/calibro/calibro-report-bg.png}
			\vfill
		}
	}
}


%opening, create title page
\title{
	Calibration report \\
	\vspace{2 mm}
	\small Automatically generated by calibro-V2.2\\
	\vspace{2 mm}
	\tiny url to be included
}
\date{\today}
\author{Calibration: simplest}
\IfFileExists{upquote.sty}{\usepackage{upquote}}{}
\begin{document}

\AddToShipoutPicture*{\BackgroundPic}

\pagestyle{fancy}
\fancyhf{}
\chead{calibration report automatically generated by calibro-V2.2}
\cfoot{url-service}
\rfoot{\thepage}

\maketitle

\begin{abstract}
This document reports the outcomes of the calibration \emph{simplest}.
Here below are listed the main results consisting of model parameters estimates and confidence intervals (Table \ref{tab:thetaES}), 
and the model prediction error, quantified by the Root Mean Squared Error (RMSE), before and after calibration (Table \ref{tab:gofES}).

% latex table generated in R 4.3.2 by xtable 1.8-4 package
% Mon Nov 17 15:53:58 2025
\begin{table}[ht]
\centering
\caption{Estimates and confidence intervals for the model parameters.} 
\label{tab:thetaES}
\begin{tabularx}{\textwidth}{X|ccc}
  \hline
PARAMETER & ESTIMATE & LOWER & UPPER \\ 
  \hline
e1\_natural\_ventilation\_rate & 1.296 & 0.860 & 1.345 \\ 
  b10\_airloophvac & 0.770 & 0.644 & 1.026 \\ 
  b12\_sat & 22.903 & 17.028 & 26.015 \\ 
  d2\_htgsp\_office\_st & 15.969 & 15.047 & 17.851 \\ 
  d1\_htgsp\_office & 23.692 & 21.187 & 23.972 \\ 
   \hline
\end{tabularx}
\end{table}
% latex table generated in R 4.3.2 by xtable 1.8-4 package
% Mon Nov 17 15:53:58 2025
\begin{table}[ht]
\centering
\caption{Goodness of fit before and after calibration.} 
\label{tab:gofES}
\begin{tabularx}{\textwidth}{X|cc}
  \hline
DATASET & BEFORE & AFTER \\ 
  \hline
data1 & 0.930 & 1.010 \\ 
   \hline
\end{tabularx}
\end{table}


The following sections contain a more comprehensive breakdown of the calibration results, 
according to the undertaken calibration steps. 
\begin{itemize}
\item Section \ref{sec:cal} depicts the main calibration results, with graphs and tables.

\end{itemize}
\end{abstract}
\newpage



%%%%%%%%%%
%DATASETS%
%%%%%%%%%%


%%%%%%%%%%%%
%ds figures%
%%%%%%%%%%%%










%%%%%%%%%%%%%%
%SESNSITIVITY%
%%%%%%%%%%%%%%


%%%%%%%%%%%%%%%%%%
%FACTOR RETENTION%
%%%%%%%%%%%%%%%%%%

%%%%%%%%%%
%TRAINING%
%%%%%%%%%%

%%%%%%%%%%%%%
%CALIBRATION%
%%%%%%%%%%%%%


	\section{Calibration results}\label{sec:cal}
	Model calibration is undertaken by exploring each parameter’s space, having boundaries as defined by the provided samples, 
	via a Markov Chain Monte Carlo algorithms, in order to infer the values reducing the discrepancy between model predictions 
	and supplied measurements. This section describes the results obtained through this process. 
	There are 4 outputs, which are listed here below.
	
\begin{itemize} 
	\item The recommended, the minimum and the maximum values for the input parameters (Table \ref{tab:theta}). The first value is the value that 
	produces the best fit to the measurement, while the second and third values quantify the residual parameter uncertainties.
	\item The probability density distribution of the calibration parameters after calibration (i.e.
	the empirical probability density distribution of the input parameters from the Markov Chain Monte Carlo sampling). Figure 
	\ref{fig:thetaplot} shows the outcome. The upper panels illustrate correlation scatter plots for pairs of parameters. 
	The red line qualitatively indicates the correlation between parameters, as flat lines indicates low level of correlation and vice-versa.
	In the diagonal panels are depicted the marginal posterior probability density distributions.
	The lower panels present kernel estimation of the posterior joint probability density distributions for pairs of model parameters.
	Clearer colors indicates higher values. 
	\item The meta-model’s Root Mean Squared Error (RMSE; an estimate of the likely error within the predictions) before calibration 
	(i.e. the RMSE between the best provided simulation and the measurements), and after calibration (i.e. the RMSE between the calibrated meta-model 
	output and the measurements). Table \ref{tab:gof} lists the outcomes.
\item The obtained fit between measured target variable and meta-model predictions. Figure \ref{fig:ds1fit} display this result.\end{itemize}

It must be kept in mind that the meta-mode is an approximation, and its output may be 
	qualitatively different from that of the real model. Therefore it is necessary to verify by feeding back 
	the recommended values into the actual model.
% latex table generated in R 4.3.2 by xtable 1.8-4 package
% Mon Nov 17 15:53:58 2025
\begin{table}[ht]
\centering
\caption{estimates and confidence intervals for the model parameters.} 
\label{tab:theta}
\begin{tabularx}{\textwidth}{X|ccc}
  \hline
PARAMETER & ESTIMATE & LOWER & UPPER \\ 
  \hline
e1\_natural\_ventilation\_rate & 1.296 & 0.860 & 1.345 \\ 
  b10\_airloophvac & 0.770 & 0.644 & 1.026 \\ 
  b12\_sat & 22.903 & 17.028 & 26.015 \\ 
  d2\_htgsp\_office\_st & 15.969 & 15.047 & 17.851 \\ 
  d1\_htgsp\_office & 23.692 & 21.187 & 23.972 \\ 
   \hline
\end{tabularx}
\end{table}
% latex table generated in R 4.3.2 by xtable 1.8-4 package
% Mon Nov 17 15:53:58 2025
\begin{table}[ht]
\centering
\caption{Goodness of fit before and after calibration.} 
\label{tab:gof}
\begin{tabularx}{\textwidth}{X|cc}
  \hline
DATASET & BEFORE & AFTER \\ 
  \hline
data1 & 0.930 & 1.010 \\ 
   \hline
\end{tabularx}
\end{table}

\begin{knitrout}
\definecolor{shadecolor}{rgb}{0.969, 0.969, 0.969}\color{fgcolor}\begin{kframe}


{\ttfamily\noindent\color{warningcolor}{\#\# Warning in par(usr): argument 1 does not name a graphical parameter}}

{\ttfamily\noindent\color{warningcolor}{\#\# Warning in par(usr): argument 1 does not name a graphical parameter}}

{\ttfamily\noindent\color{warningcolor}{\#\# Warning in par(usr): argument 1 does not name a graphical parameter}}

{\ttfamily\noindent\color{warningcolor}{\#\# Warning in par(usr): argument 1 does not name a graphical parameter}}

{\ttfamily\noindent\color{warningcolor}{\#\# Warning in par(usr): argument 1 does not name a graphical parameter}}\end{kframe}\begin{figure}[!h]
\includegraphics[width=\maxwidth]{figure/thetaplot-1} \caption[calibration parameters posterior probability density distributions]{calibration parameters posterior probability density distributions}\label{fig:thetaplot}
\end{figure}

\end{knitrout}
\begin{knitrout}
\definecolor{shadecolor}{rgb}{0.969, 0.969, 0.969}\color{fgcolor}\begin{figure}[!h]
\includegraphics[width=\maxwidth]{figure/ds1fit-1} \caption[achieved fit for dataset data1]{achieved fit for dataset data1}\label{fig:ds1fit}
\end{figure}

\end{knitrout}









\end{document}
